% !TEX root = ../notes_template.tex
\chapter{Linear Transformations}\label{chp:lintrans}

% \minitoc

% Inset a quote for the chapter
\begin{flushright}
\textit{``Though a bit of an exaggeration, it can be said that a mathematical problem can be solved only if it can be reduced to a calculation in linear algebra. And a calculation in linear algebra will reduce ultimately to the solving of a system of linear equations, which in turn comes down to the manipulation of matrices.''}\\
% Draw a small line at the end of the quote
\rule{0.5\textwidth}{.4pt}\\
\textbf{Thomas A Garrity} \small{in \textit{All the Mathematics You Missed.}}
\end{flushright}

I concede that the first two chapters were a bit dry. Hopefully, you will find topics from now on a bit more interest. Linear algebra is the study of linear transformations. We already saw an example of a linear transformation in Section~\ref{sec:ch01-lin-func}. We will look at linear transformations in in their general form and see how matrices can be used to represent and understand them.

\section{What is a linear transformation?}\label{sec:ch03-lin-trans-def}


\section{Matrices represent linear transformations}\label{sec:ch03-mat-lin-trans}